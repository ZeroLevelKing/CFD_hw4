\documentclass[UTF8]{ctexart}
\usepackage{amsmath}   
\usepackage{booktabs}  
\usepackage{geometry}  
\usepackage{hyperref}
\usepackage{graphicx} 
\usepackage{float}
\graphicspath{{figure/}} % 指定放置图片的子文件夹路径
\geometry{a4paper, left=2.5cm, right=2.5cm, top=2.5cm, bottom=2.5cm}

\begin{document}

\title{计算流体力学第四次作业}
\author{朱林-2200011028}
\date{\today}
\maketitle


\section{数理算法原理}

\subsection{控制方程与边界条件}
二维稳态温度场满足拉普拉斯方程:
\begin{equation}
    \frac{\partial^2 T}{\partial x^2} + \frac{\partial^2 T}{\partial y^2} = 0
\end{equation}

\textbf{边界条件}:
\begin{itemize}
    \item 上边界 ($ y=12\text{cm} $):$ T = 100^\circ \text{C} $ 
    \item 左边界 ($ x=0 $)、右边界 ($ x=15\text{cm} $)、下边界 ($ y=0 $):$ T = 20^\circ \text{C} $
\end{itemize}

\subsection{有限差分离散化}
\subsubsection{网格参数}
\begin{align}
    \Delta x &= \frac{15}{N_x-1}, \quad x_i = (i-1)\Delta x \quad (i=1,2,...,N_x) \\
    \Delta y &= \frac{12}{N_y-1}, \quad y_j = (j-1)\Delta y \quad (j=1,2,...,N_y)
\end{align}

\subsubsection{离散方程}
内部节点 ($ 2 \leq i \leq N_x-1, 2 \leq j \leq N_y-1 $):
\begin{equation}
    \frac{T_{i+1,j} - 2T_{i,j} + T_{i-1,j}}{(\Delta x)^2} + \frac{T_{i,j+1} - 2T_{i,j} + T_{i,j-1}}{(\Delta y)^2} = 0
\end{equation}

均匀网格 ($ \Delta x = \Delta y = h $) 简化为:
\begin{equation}
    T_{i,j} = \frac{1}{4}\left( T_{i+1,j} + T_{i-1,j} + T_{i,j+1} + T_{i,j-1} \right)
\end{equation}

\subsubsection{边界节点处理}
\begin{align}
    T_{1,j} &= 20 \quad (\forall j),\ \ T_{N_x,j} = 20 \quad (\forall j) \\
    T_{i,1} &= 20 \quad (\forall i),\ \ T_{i,N_y} = 100 \quad (\forall i)
\end{align}

\subsection{迭代算法修正}
\subsubsection{高斯-赛德尔迭代}
按列优先顺序更新 ($ j $ 从 2 到 $ N_y-1 $):
\begin{equation}
    T_{i,j}^{(k+1)} = \frac{1}{4}\left( T_{i-1,j}^{(k+1)} + T_{i+1,j}^{(k)} + T_{i,j-1}^{(k+1)} + \underbrace{T_{i,j+1}^{(k)}}_{\text{当 } j+1=N_y \text{ 时取 } 100} \right)
\end{equation}

\subsubsection{SOR 加速算法}
引入松弛因子 $ \omega $:
\begin{equation}
    T_{i,j}^{(k+1)} = (1-\omega)T_{i,j}^{(k)} + \frac{\omega}{4}\left( T_{i-1,j}^{(k+1)} + T_{i+1,j}^{(k)} + T_{i,j-1}^{(k+1)} + T_{i,j+1}^{(k)} \right)
\end{equation}

\subsection{收敛性分析}
\begin{itemize}
    \item \textbf{残差定义}:$ R^{(k)} = \max\limits_{\substack{2 \leq i \leq N_x-1 \\ 2 \leq j \leq N_y-1}} |T_{i,j}^{(k+1)} - T_{i,j}^{(k)}| $
    \item \textbf{收敛判据}:$ R^{(k)} < \epsilon \quad (\text{通常取 } \epsilon = 10^{-5}) $
    \item \textbf{最优松弛因子}:
    \begin{equation}
        \omega_{\text{opt}} = \frac{2}{1 + \sin\left( \frac{\pi}{\max(N_x,N_y)-1} \right)}
    \end{equation}
\end{itemize}

\subsection{数值实现伪代码}
\begin{verbatim}
# 初始化温度场
T = 20 * np.ones((Nx, Ny))
T[:, -1] = 100  # 设置上边界

for k in range(max_iter):
    R = 0
    for j in range(1, Ny-1):
        for i in range(1, Nx-1):
            T_new = 0.25 * (T[i+1,j] + T[i-1,j] 
                   + T[i,j+1] + T[i,j-1])
            R = max(R, abs(T_new - T[i,j]))
            T[i,j] = T_new
    if R < epsilon:
        break
\end{verbatim}

\subsection{理论验证}
\begin{itemize}
    \item \textbf{对称性检验}:温度场应关于 $ x=7.5\text{cm} $ 对称
    \item \textbf{极值原理}:内部温度 $ T \in (20,100)^\circ \text{C} $ \\
    \item \textbf{能量守恒}:总热流量进出平衡
\end{itemize}

\end{document}
%附录
\newpage
\appendix
\section{AI工具使用声明表}
\begin{table}[H]
    \centering
    \begin{tabular}{c|c|c}
        \hline
        使用内容 & 工具名称 & 使用目的 \\ \hline
        hw3.tex 1-9行、图片插入 & Github Copilot & 调整pdf格式,调用宏包,省略插入图片的重复性工作 \\ 
        .gitignore & Github Copilot & 针对于python和latex的.gitignore文件,完全由Copilot生成  \\
        main.py 部分matplotlib部分 & Github Copilot & 省略图片绘制的重复性工作
    \end{tabular}
    \label{tab:AI_tools}
\end{table}
\end{document}